%%
%% In this note, we explain how to generate the low data in our paper
%% ``Holonomic gradient method for distribution function of 
%%   a weighted sum of noncentral chi-square random variables''
%%
%% To compile this TeX file, simply type the following command:
%%     latex note-en.tex && latex note-en.tex
%%

\iffalse
%\iftrue

\documentclass{article}

\else 

\documentclass[12pt]{article}

\textwidth 155 true mm
\textheight 247 truemm
\topmargin -0.4 true mm
\headheight 0 true mm
\headsep 0 true mm
\oddsidemargin 2.1 true mm
\evensidemargin 2.1 true mm
\footskip 5 true mm
%\titlesep 12pt

\fi

%\pagestyle{empty}

\usepackage{color}
%\usepackage{refcheck}
\usepackage{url}
%\usepackage{amssymb}
%\usepackage{amsmath}
%\usepackage{graphicx}
\usepackage{listings}

%\usepackage{amsthm}
%\newtheorem{theorem}{Theorem}
%\newtheorem{proposition}{Proposition}
%\newtheorem{lemma}{Lemma}
%\newtheorem{corollary}{Corollary}
%\newtheorem{conjecture}{Conjecture}
%\newtheorem{algorithm}{Algorithm}
%\theoremstyle{definition}
%\newtheorem{definition}{Definition}
%\newtheorem{example}{Example}

%% for shell script
\lstdefinestyle{BashInputStyle}{
  language=bash,
  firstline=2,% Supress the first line that begins with `%`
  basicstyle=\small\sffamily,
  numbers=left,
  numberstyle=\tiny,
  numbersep=3pt,
  frame=tb,
  columns=fullflexible,
  backgroundcolor=\color{white},
  linewidth=0.9\linewidth,
  xleftmargin=0.1\linewidth
}

\def\pd#1{\partial_{#1}}

\title{A Note for Numerical Experiments}
\author{Tamio Koyama}
\date{March 16, 2015}

\begin{document}
\maketitle
%\thispagestyle{empty}

\begin{abstract}
In this note, we explain how to generate the low data in our paper
\cite{koyama-takemura2}.
%,T.~Koyama, A.~Takemura, 
%``Holonomic gradient method for distribution function of 
%a weighted sum of noncentral chi-square random variables'',
%\url{http://arxiv.org/abs/1503.00378}.
\end{abstract}

\section{Preparation}
For our numerical experiments, we prepared C program {\tt program.c}, 
shell scripts {\tt tab1and2.sh}, {\tt tab3.sh},
and {\tt R} scripts {\tt fig1.R}, {\tt fig2.R}, {\tt tab1and2.R}, {\tt tab3.R}.
Our computer system is linux(debian), and we utilized GNU scientific library
BLAS, and {\tt R}.
The source file {\tt program.c} can be compiled by the following command:
\begin{lstlisting}[style=BashInputStyle]

gcc program.c -lm -lgsl -lblas -O0 -Wall 
\end{lstlisting}

 
In our numerical experiments, we generated the following files:
{\tt fig1.txt}, {\tt fig1.eps},
{\tt fig2.txt}, {\tt fig2-1.eps},{\tt fig2-2.eps},
{\tt tab1and2.txt},{\tt tab1and2.tex},
{\tt tab3.txt},{\tt tab3.tex},{\tt tab3.eps}.

\section{How to Generate Low Data}
%In this section, we explain how to generate Figure 1 in \cite{koyama-takemura2}.
\noindent{\bf Figure 1}\/.
The following command generates the low data {\tt fig1.txt}
and eps file {\tt fig1.eps} for figure 1:
\begin{lstlisting}[style=BashInputStyle]

./a.out 2 3 20.0 3.0 2.0 1.0 0.01 0.02 0.03 > fig1.txt
R -f fig1.R
\end{lstlisting}

\medskip\noindent{\bf Figure 2}\/.
The following command generates the low data {\tt fig2.txt}
and eps files {\tt fig2-1.eps} and {\tt fig2-2.eps} for figure 2:
\begin{lstlisting}[style=BashInputStyle]

./a.out 11 3 20.0 3.0 2.0 1.0 0.01 0.02 0.03 > fig2.txt
R -f fig2.R
\end{lstlisting}

\medskip\noindent{\bf Table 1 and Table 2}\/.
Run the shell script {\tt tab1and2.sh}, then the low data {\tt tab1and2.txt} and 
the \TeX file {\tt tab1and2.tex} are generated.

\medskip\noindent{\bf Table 3 and its figure}\/.
Run the shell script {\tt tab3.sh}, then the low data {\tt tab3.txt}, 
the TeX source {\tt tab3.tex}, and the eps file {\tt tab3.eps} are generated.


%\iffalse
\iftrue

\begin{thebibliography}{1}

\bibitem{koyama-takemura2}
T.~Koyama and A.~Takemura.
\newblock Holonomic gradient method for distribution function of a weighted sum
  of noncentral chi-square random variables.
\newblock \url{http://arxiv.org/abs/1503.00378}, 2015.

\end{thebibliography}

\else

\bibliographystyle{plain}
\bibliography{reference}

\fi

%\appendix
\end{document}
